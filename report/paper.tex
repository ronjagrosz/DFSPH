\documentclass[journal]{vgtc}                % final (journal style)
%\documentclass[review,journal]{vgtc}         % review (journal style)
%\documentclass[widereview]{vgtc}             % wide-spaced review
%\documentclass[preprint,journal]{vgtc}       % preprint (journal style)
%\documentclass[electronic,journal]{vgtc}     % electronic version, journal

%% Uncomment one of the lines above depending on where your paper is
%% in the conference process. ``review'' and ``widereview'' are for review
%% submission, ``preprint'' is for pre-publication, and the final version
%% doesn't use a specific qualifier. Further, ``electronic'' includes
%% hyperreferences for more convenient online viewing.

%% Please use one of the ``review'' options in combination with the
%% assigned online id (see below) ONLY if your paper uses a double blind
%% review process. Some conferences, like IEEE Vis and InfoVis, have NOT
%% in the past.

%% Please note that the use of figures other than the optional teaser is not permitted on the first page
%% of the journal version.  Figures should begin on the second page and be
%% in CMYK or Grey scale format, otherwise, colour shifting may occur
%% during the printing process.  Papers submitted with figures other than the optional teaser on the
%% first page will be refused.

%% These three lines bring in essential packages: ``mathptmx'' for Type 1
%% typefaces, ``graphicx'' for inclusion of EPS figures. and ``times''
%% for proper handling of the times font family.

\usepackage{mathptmx}
\usepackage{graphicx}
\usepackage{times}
\usepackage{balance}
\usepackage[nooneline,hang,it,IT]{subfigure}

%% We encourage the use of mathptmx for consistent usage of times font
%% throughout the proceedings. However, if you encounter conflicts
%% with other math-related packages, you may want to disable it.

%% This turns references into clickable hyperlinks.
\usepackage[bookmarks,backref=true,linkcolor=black]{hyperref} %,colorlinks
\hypersetup{
  pdfauthor = {},
  pdftitle = {},
  pdfsubject = {},
  pdfkeywords = {},
  colorlinks=true,
  linkcolor= black,
  citecolor= black,
  pageanchor=true,
  urlcolor = black,
  plainpages = false,
  linktocpage
}

%% If you are submitting a paper to a conference for review with a double
%% blind reviewing process, please replace the value ``0'' below with your
%% OnlineID. Otherwise, you may safely leave it at ``0''.
\onlineid{0}

%% declare the category of your paper, only shown in review mode
\vgtccategory{Research}

%% allow for this line if you want the electronic option to work properly
\vgtcinsertpkg

%% In preprint mode you may define your own headline.
%\preprinttext{To appear in an IEEE VGTC sponsored conference.}

%% Paper title.

\title{\LARGE TN1008 \\ Advanced Simulation and Visualization of Fluids in Computer Graphics \\ \Large Divergence-Free Smoothed Particle Hydrodynamics}

%% This is how authors are specified in the journal style

%% indicate IEEE Member or Student Member in form indicated below
\author{Ronja Grosz, rongr946\\ Isabell Jansson, isaja187\\ Jonathan Bosson, jonbo665}
\date{\today}
%\authorfooter{
%% insert punctuation at end of each item
%\item
 %Ronja Grosz is a student at Link\"oping University, Sweden, e-mail: rongr946@student.liu.se.
%}

%% Abstract section.
\abstract{ 
}
%% Keywords that describe your work. Will show as 'Index Terms' in journal
%% please capitalize first letter and insert punctuation after last keyword
\keywords{Divergence-free, SPH, divergence correction, density correction.}



%%%%%%%%%%%%%%%%%%%%%%%%%%%%%%%%%%%%%%%%%%%%%%%%%%%%%%%%%%%%%%%%
%%%%%%%%%%%%%%%%%%%%%% START OF THE PAPER %%%%%%%%%%%%%%%%%%%%%%
%%%%%%%%%%%%%%%%%%%%%%%%%%%%%%%%%%%%%%%%%%%%%%%%%%%%%%%%%%%%%%%%%

\begin{document}

\firstsection{Introduction}
\maketitle 
%explain the context of the work:
	%What exactly is the problem?
	%What have you created?

\section{Background and Related Work}
%What have people done before?
	%When addressing this problem, when addressing similar problems
%What have you done that makes you approach different?
Based on~\cite{bender}

\section{Method}
%What have you done?
%How did you do it?
%Why did you do it that way?
%May want to mention why you *didn’t* do it in some 
%other possible way
%Clear enough that you could hand the report to another student and they could reproduce the work!
%Probably no need to mention programming language etc
\subsection{Neighbourhood search}
\subsection{Divergence solver}
\subsection{Density solver}
\subsection{Kernel}
\subsection{Navier-stokes}
% Write about non pressure forces, velocity osv..?
\subsection{Adapted time step}
\subsection{Density and alpha factors}

\section{Implementation}
%Details of how it was done
%What hardware
%What software
%Why did you do it that way?
%What limitations, if any, did this present?
%*No code*!!!!!
 
    % Talk about what we used for plugins, computers etc maybe or should this be in method?

\section{Results}
%What is the end product:
%What does it look like?
%How does one control and interact with it?
%How quick is it?
%What limitations does it have?
%Be self-critical and honest about it - or I will be



\section{Conclusions and Future Work}
%Based on the results and evaluation
%Say what you’ve done right... ...and what you’ve done wrong! - be honest!
%Suggest some possible ways that the work could have been done better
%Suggest some ways that it could be extended and improved by adding more effort

% Use ghost-SPH or similar

\bibliographystyle{abbrv}
\bibliography{./refs}

\end{document}
\grid
